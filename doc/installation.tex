\documentclass[11pt,final]{amsart}
\usepackage[margin=1in]{geometry}
\usepackage{underscore}

\usepackage{booktabs}           % better rules in tables

\newcommand{\PISMREV}{\input{revision.tex}}
\newcommand{\PETSCREL}{3.1}

\addtolength\topmargin{+.3in}
\addtolength\textheight{-0.2in}
%\addtolength{\oddsidemargin}{-.4in}
%\addtolength{\evensidemargin}{-.4in}
%\addtolength{\textwidth}{0.9in}
\newcommand{\normalspacing}{\renewcommand{\baselinestretch}{1.1}\tiny\normalsize}
\newcommand{\tablespacing}{\renewcommand{\baselinestretch}{1.0}\tiny\normalsize}
\normalspacing

\usepackage{bm,url,xspace,verbatim}
\usepackage{amssymb,amsmath}
\usepackage[final,pdftex]{graphicx}

\usepackage[pdftex,
                colorlinks=true,
                plainpages=false, % only if colorlinks=true
                linkcolor=blue,   % only if colorlinks=true
                citecolor=blue,   % only if colorlinks=true
                urlcolor=blue     % only if colorlinks=true
]{hyperref}[2010/03/30 v6.80u Hypertext links for LaTeX]

%% uncomment to see locations of index entries
%\usepackage{showidx}

\newcommand{\ddt}[1]{\ensuremath{\frac{\partial #1}{\partial t}}}
\newcommand{\ddx}[1]{\ensuremath{\frac{\partial #1}{\partial x}}}
\newcommand{\ddy}[1]{\ensuremath{\frac{\partial #1}{\partial y}}}
\renewcommand{\t}[1]{\texttt{#1}}
\newcommand{\Matlab}{\textsc{Matlab}\xspace}
\newcommand{\bq}{\mathbf{q}}
\newcommand{\bU}{\mathbf{U}}
\newcommand{\eps}{\epsilon}
\newcommand{\grad}{\nabla}
\newcommand{\Div}{\nabla\cdot}

%% macros having to do with documentation for options; note these appear in the index
\newcommand{\und}{\_\!\_}

%\newcommand{\alphaoptionindex}[1]{\index{alphabetical list of options for PISM (and PETSc)!\texttt{-#1}}}
\newcommand{\pismoptionindex}[1]{\index{options for PISM (and PETSc)!\texttt{-#1}}}
\newcommand{\intextoption}[1]{\texttt{-#1}\pismoptionindex{#1}}

%\newcommand{\rawopt}[1]{\vspace{1mm}\noindent \large\texttt{-#1}\normalsize\pismoptionindex{#1}\alphaoptionindex{#1}}
\newcommand{\rawopt}[1]{\vspace{1mm}\noindent \Large\texttt{-#1}\normalsize\pismoptionindex{#1}}
\newcommand{\opt}[1]{\rawopt{#1}\,:\quad}
\newcommand{\optdef}[2]{\rawopt{#1}\,[\textsl{#2}]:\quad}
\newcommand{\optrestrict}[2]{\rawopt{#1}\,[\texttt{#2} \textsl{only}]:\quad}
\newcommand{\optdefrestrict}[3]{\rawopt{#1}\,[\textsl{#2}]\,[\texttt{#3} \textsl{only}]:\quad}


% preamble:

\makeindex

\title[PISM Installation Manual]{\protect{\Large \emph{PISM}, a Parallel Ice
    Sheet Model:\normalsize} \\ \protect{\Large \bigskip \bigskip Installation
    Manual\normalsize}}

\author[]{the PISM authors}

\date{\today.  Get PISM development version by \\
\phantom{foobar} \qquad\texttt{svn co http://svn.gna.org/svn/pism/trunk pism-dev} \\
Based on PISM revision \PISMREV\quad and PETSC release \PETSCREL.}


\pdfinfo{
/Title (PISM Installation Manual)
/Author (the PISM authors)
/Subject (Downloading and installing PISM, a Parallel Ice-Sheet Model)
/Keywords (PISM ice-sheet modeling installation)
}

\sloppy

\begin{document}
\maketitle
\thispagestyle{empty}

\vspace{1.5in}
\setcounter{tocdepth}{2}
\tableofcontents



\newpage
\section*{Introduction}

\large
This \emph{Installation Manual} describes how to download the PISM source code and install PISM and the libraries it needs.  Information about PISM, including a \emph{User's Manual}, is on-line at
\bigskip
\begin{center}
  \href{http://www.pism-docs.org}{\t{www.pism-docs.org}}
\end{center}
\bigskip
\noindent The fastest path to a fully functional PISM installation is to use a Linux system with a Debian-based package system (e.g.~Ubuntu) and then install Python packages following section \ref{sec:python}.  See subsection \ref{subsec:debian} about the Debian installation route.

\bigskip
See the \emph{User's Manual} for authorship of PISM.
\vfill

\large
\begin{center}
\parbox{5.5in}{ \emph{WARNING}:\index{PISM!warning}  PISM is an ongoing project.  Ice sheet modeling is complicated and is generally not mature.  Please don't trust the results of PISM or any other ice sheet model without a fair amount of exploration.

\bigskip
Also, please don't expect all your questions to be answered here.  Write to us with questions at \href{mailto:help@pism-docs.org}{\texttt{help@pism-docs.org}}.}
\normalsize
\end{center}
\normalsize
\vfill

\begin{quote}
\textsl{Copyright (C) 2004--2011 the PISM authors.}
\medskip

\noindent \textsl{This file is part of PISM.  PISM is free software; you can redistribute it and/or modify it under the terms of the GNU General Public
  License as published by the Free Software Foundation; either version 2 of the License, or (at your option) any later version.  PISM is distributed in the hope that it will be useful, but WITHOUT ANY WARRANTY; without even the implied
  warranty of MERCHANTABILITY or FITNESS FOR A PARTICULAR PURPOSE. See the GNU General Public License for more details.  You should have received a copy of the GNU General Public License\index{GPL (\emph{GNU Public License})} along with PISM. If not, write to the Free Software Foundation, Inc., 51 Franklin St, Fifth Floor, Boston, MA 02110-1301 USA}
\end{quote}


\clearpage
\section{Libraries and programs needed by PISM}
\label{sec:prerequisites}

\bigskip
\normalspacing
This table lists required dependencies for PISM alphabetically.
\bigskip
\newcommand{\fattablespacing}{\renewcommand{\baselinestretch}{1.5}\tiny\normalsize}

\fattablespacing
\newcommand\pairstack[2]{#1 \quad (\small#2\normalsize)}
\begin{center}
\begin{tabular*}{1.0\linewidth}{lp{0.5\linewidth}}\toprule
  \textbf{Required Program or Library} & \textbf{Comment} \\
  \midrule
  \pairstack{GSL}{\href{http://www.gnu.org/software/gsl/}{\t{www.gnu.org/software/gsl}}}  &  \\
  \pairstack{MPI}{\href{http://www-unix.mcs.anl.gov/mpi/}{\t{www-unix.mcs.anl.gov/mpi}}}  & \\
  \pairstack{NetCDF}{\href{http://www.unidata.ucar.edu/software/netcdf/}{\t{www.unidata.ucar.edu/software/netcdf}}}  & version $\ge$ 3.6.1\\
  \pairstack{PETSc}{\href{http://www-unix.mcs.anl.gov/petsc/petsc-as/}{\t{www-unix.mcs.anl.gov/petsc}}}  & version $\ge$ \PETSCREL \\
  \bottomrule
\end{tabular*}
\end{center}

\normalspacing
\bigskip

Before installing these ``by hand'', check the Debian and Mac OS X sections below for specific how-to.  In particular, if multiple MPI implementations (e.g.~MPICH and OPEN-MPI) are installed then PETSc can under some situations ``get confused'' and throw MPI-related errors.  Even package systems have been known to allow this confusion.

One library is only used for the earth deformation (``isostatic rebound'') model, and is thus only recommended:
\bigskip

\fattablespacing
\begin{center}
\begin{tabular*}{1.0\linewidth}{ll}\toprule
  \textbf{Recommended Library} & \textbf{Comment} \\
  \midrule
  \pairstack{FFTW}{\href{http://www.fftw.org/}{\t{www.fftw.org}}}\hspace{1.9in} & \\
  \bottomrule
\end{tabular*}
\end{center}

\normalspacing

\vspace{0.2in}

\bigskip


In addition, Python (\url{http://python.org/}) is needed both in the PETSc installation process and in scripts related to PISM pre- and post-processing, while Subversion (\url{http://subversion.apache.org/http://subversion.tigris.org/}) is usually needed to download the PISM code.  Both should be included in any Linux/Unix distribution.  Additionally, the following Python packages are needed to do all the examples in the \emph{User's Manual} (which run python scripts).
\bigskip

\fattablespacing
\begin{center}
\begin{tabular*}{1.0\linewidth}{lp{0.35\linewidth}}\toprule
  \textbf{Recommended Python Package} & \textbf{Importance} \\
  \midrule
  \pairstack{\t{matplotlib}}{\href{http://matplotlib.sourceforge.net/}{\t{matplotlib.sourceforge.net}}} & used in some scripts \\
  \pairstack{\t{netcdf4-python}}{\href{http://code.google.com/p/netcdf4-python/}{\t{code.google.com/p/netcdf4-python}}}  & used in \emph{most} scripts  \\    
  \pairstack{\t{numpy}}{\href{http://numpy.scipy.org/}{\t{numpy.scipy.org}}} & used in \emph{most} scripts \\
  \pairstack{\t{scikits.delaunay}}{\href{http://scipy.org/scipy/scikits}{\t{scipy.org/scipy/scikits}}} & used in few scripts \\
  \bottomrule
\end{tabular*}
\end{center}
\normalspacing


\newpage
\section{Installation Cookbook}\label{sec:cookbook}

\subsection{Installing PISM on a Debian-based Linux system} \label{subsec:debian}

\subsubsection{Installing PISM using a binary package}
\label{sec:deb-pism-binary}

If you are not planning on using a development version of PISM and don't need to extend PISM by writing new code, you might be
able to install it with just a few clicks.

Download a binary package from \url{http://www.pism-docs.org} and either run
\begin{verbatim}
$ sudo dpkg -i pism_0.4-1.deb
\end{verbatim}%$
or double-click on the file and click ``Install'' (will require administrative privileges). This will install PISM, making sure that
all the libraries it needs are in place and downloading them if necessary.

This route is the fastest and easiest if you want to learn about or use a stable PISM version. If you do need features only
present in development versions, you will need to compile PISM using its source code.

\subsubsection{Installing PISM's prerequisites using a meta-package and building PISM from source}
\label{sec:deb-pism-toolkit}

If you are using a Debian-based system such as Ubuntu but need to compile PISM from source, you can download and install the
``pism toolkit'' meta-package.

This package does \emph{not} install PISM; it tells the package manager to install all the libraries and tools needed to compile
PISM from sources.

To use this route of installing PISM's prerequisites, download \texttt{pism-toolkit_0.1-1.deb} from \url{http://www.pism-docs.org} and run
\begin{verbatim}
$ sudo dpkg -i pism-toolkit_0.1-1.deb
\end{verbatim}%$
or double-click on the file and click ``Install''. Once this is done, see section \ref{sec:install-cmake} for configuration and
compilation instructions.

\subsubsection{Installing PISM's prerequisites if a meta-package does not work}
\label{sec:deb-libraries-by-hand}

If you need to compile PISM from sources and you can not use the meta-package described in the previous section, you can use your
package manager to manually select packages PISM requires.

Find and install (using \texttt{apt-get} or \texttt{synaptic}, for example) the following packages:
  \begin{center}
    \begin{tabular*}{0.9\linewidth}{p{0.2\linewidth}p{0.7\linewidth}}
      \toprule
      \emph{Package name} & \emph{Comments}\\
      \midrule
      \texttt{petsc-dev} & depends on \texttt{libpetsc3.x.x-dev}; this package 
      depends on \texttt{libpetsc3.x.x}, \texttt{libopenmpi-dev}, \texttt{libx11-dev},
      \texttt{libblas-dev}, \texttt{liblapack-dev}, \texttt{gfortran}, and others; 
      note \texttt{libopenmpi-dev} depends on \texttt{libopenmpi1}\\
      \texttt{libgsl0-dev} & \\
      \texttt{libfftw3-dev} & \\
      \texttt{netcdf-bin} & (and perhaps grab \texttt{nco} at the same time)\\
      \texttt{libnetcdf-dev} & \\
      \texttt{g++} & without this you may see a compiler error: ``\texttt{gcc: error} \texttt{trying to} \texttt{exec 'cc1plus'}''\\
      \texttt{subversion} & \\
      \texttt{python-dev} & (helps with scripts \dots perhaps not essential) \\
      \texttt{ncview} & (grab if available) \\
      \bottomrule
    \end{tabular*}
  \end{center}

See section \ref{sec:install-cmake} for configuration and compilation instructions and refer to the \emph{Getting Started} section
of the \emph{User's Manual} once you are done.

\vspace{0.3in}

\subsubsection*{NCVIEW}  Depending on the age of your Debian-based Linux, there may not be a \texttt{ncview} Debian package.  The following sequence has worked with minimal pain for installation from the \texttt{ncview} homepage:

Grab latest source from
\begin{center}
  \url{http://meteora.ucsd.edu/~pierce/ncview_home_page.html}
\end{center}
Basically just get the \texttt{.tar.gz} and follow install instructions in top-level ``\texttt{README}''.  An error on a Debian system about missing \texttt{X} headers etc may be resolved by getting the \texttt{xorg-dev} package.  More generally, one may tell the \texttt{ncview} \texttt{configure} script where to find \texttt{X}:
\begin{verbatim}
$  ./configure --x-libraries=/usr/lib
\end{verbatim}

\vspace{0.3in}

\newpage
\subsection{Generic PISM installation, using PETSc source code}\label{subsec:generic}

\subsubsection{Installing prerequisites}
\renewcommand{\labelenumi}{\textbf{\arabic{enumi}.}~}
\begin{enumerate}
\item You will need a UNIX system with Internet access. A GNU/Linux environment will be easiest but other UNIX versions have been
  used successfully. Package management systems are useful for installing many of the tools listed in section
  \ref{sec:prerequisites}, \emph{but} an up-to-date PETSc distribution may not be available in your system's package
  repositories. You will need \href{http://www.python.org/}{Python} and \href{http://subversion.tigris.org/}{Subversion}
  installed, but these are included in all current Linux distributions. To use the (recommended) graphical output of PISM you will
  need an \href{http://www.x.org/}{X Windows server}.

\item As PISM is currently under rapid development, we distribute it only as compilable source code. On the other hand, there are
  several software libraries needed by PISM. Therefore the ``header files'' for these libraries are required for building PISM. In
  particular this means that the ``developer's versions'' of the libraries are needed if the libraries are downloaded from package
  repositories like Debian.

\item PISM uses \href{http://www.unidata.ucar.edu/software/netcdf/}{NetCDF (= \emph{network Common Data Form})}\index{NetCDF} for
  an input and output file format. If it is not already present, install it using the instructions at the web-page or using a
  package management system.

\item PISM uses the \href{http://www.gnu.org/software/gsl/}{GSL (= \emph{GNU Scientific Library})}\index{GSL (\emph{GNU Scientific
      Library})} for certain numerical calculations and special functions. If it is not already present, install it using the
  instructions at the web-page or using a package management system.

\item PISM optionally uses the \href{http://www.fftw.org/}{``FFTW'' library (FFTW = \emph{Fastest Fourier Transform in the
      West})}\index{FFTW (\emph{Fastest Fourier Transform in the West})} in one approximation of the deformation of the solid earth (bed) under
  ice loads.\footnote{Bueler and others (2007). \emph{Fast computation of a viscoelastic deformable Earth model for ice sheet simulation}, Ann.~Glaciol.~46, pp.~97–-105.}  If you want the functionality of this bed deformation model, which is coupled to the ice flow and which we
  recommend, install FFTW or check that it is installed already. If this library is absent, all of PISM will work
  \emph{except} for this bed deformation model.

\item You will need a version of \href{http://www-unix.mcs.anl.gov/mpi/}{MPI (= \emph{Message Passing Interface})}.\index{MPI
    (\emph{Message Passing Interface})} Your system may have an existing MPI installation, in which case the path to the MPI
  directory will be used when installing PETSc below. Otherwise we recommend that you allow PETSc to download
  \href{http://www-unix.mcs.anl.gov/mpi/mpich2/}{MPICH2} as part of the PETSc configure process (next). In either case, once MPI
  is installed, you will want to add the MPI \texttt{bin} directory to your path so that you can invoke MPI using the \texttt{mpiexec}
  or \texttt{mpirun} command. For example, you can add it with the statement

\texttt{export PATH=/home/user/mympi/bin:\$PATH}  \qquad (for \texttt{bash} shell)

\noindent or

\texttt{setenv PATH /home/user/mympi/bin:\$PATH}  \qquad (for \texttt{csh} or \texttt{tcsh} shell).

\noindent Such a statement can, of course, appear in your \texttt{.bashrc}, \texttt{.profile}, or \texttt{.cshrc} file so that there is
no need to retype it each time you use MPI.

\medskip
\begin{center}
  \emph{From now on this manual will assume the use of} bash.
\end{center}
\medskip

\item PISM uses \href{http://www-unix.mcs.anl.gov/petsc/}{PETSc (= \emph{Portable Extensible Toolkit for
 Scientific Computation})}.\index{PETSc (\emph{Portable Extensible Toolkit for Scientific computation})}  Let's get this out of the way: note ``PETSc'' is pronounced ``pet-see''.  Download the PETSc source by grabbing the current gzipped tarball:
\begin{center}
    \href{http://www-unix.mcs.anl.gov/petsc/}{\t{www-unix.mcs.anl.gov/petsc/}}
\end{center}
PISM requires a version of PETSc which is \texttt{\PETSCREL} or later.  The ``lite'' form of PETSc is fine if you are willing to depend on an Internet connection for accessing the PETSc documentation.

You should configure and build PETSc \emph{essentially} as described on the PETSc installation page, but it might be best to read the following comments on the PETSc configure and build process first:

\renewcommand{\labelenumii}{(\roman{enumii})}\begin{enumerate}
\item Untar in your preferred location, but note PETSc should \emph{not} be configured (next) using root privileges.  Note that you will need to define the environment variable \texttt{PETSC_DIR}\index{PETSC\und DIR} before configuring PETSc (next).  For instance, once you have entered the PETSc directory just untarred, \texttt{export PETSC_DIR=\$PWD}. (Or \texttt{PETSC_DIR=`pwd`}.  In this case, note the use of the backprime (\emph{accent-grave}) character, and not the single apostrophe \texttt{'}.)

\item When you run the configure script in the PETSc directory, the following options are recommended; note PISM uses shared libraries by default:\index{PETSC\und ARCH}
\begin{verbatim}
$  export PETSC_DIR=$PWD
$  export PETSC_ARCH=linux-gnu-opt
$  ./config/configure.py --with-shared --with-dynamic --with-debugging=no
\end{verbatim}

Turning off the inclusion of the debugging symbols gives a significant speed improvement.

Using shared and dynamic libraries may be unwise on certain clusters, etc.; check with your system administrator.

Note that there is no PISM use of Fortran, and that it is sometimes convenient to have PETSc grab a local copy of BLAS\index{BLAS (\emph{Basic Linear Algebra Subsystem})} and LAPACK\index{LAPACK (\emph{Linear Algebra PACKage})} rather than using the system-wide version.  So one may add ``\texttt{--with-fortran=0} \texttt{--download-c-blas-lapack=1}'' to the other configure options.

\item If there is an existing MPI\index{MPI (\emph{Message Passing Interface})} installation, for example at \texttt{/home/user/mympi/}, one can point PETSc to it by adding the option \texttt{--with-mpi-dir=/home/user/mympi/}.  The path used in this option must have MPI executables \texttt{mpicxx} and \texttt{mpicc}, and either \texttt{mpiexec} or \texttt{mpirun}, in sub-directory \texttt{bin/} and MPI library files in sub-directory \texttt{lib/}.

\item On the other hand, it seems common that one needs to tell PETSc to download MPI\index{MPI (\emph{Message Passing Interface})} into a place it understands, even if there is an existing MPI.  If you get messages suggesting that PETSc cannot configure using your existing MPI, you might try \texttt{configure.py} with option \texttt{--download-mpich=1}.

\item Configuration of PETSc for a batch system requires special procedures described at the PETSc documentation site.  One starts with a configure option \texttt{--with-batch=1}.  See the ``Installing on machine requiring cross compiler or a job scheduler'' section of the \href{http://www-unix.mcs.anl.gov/petsc/petsc-2/documentation/installation.html}{PETSc installation page}.

\item  Configuring PETSc takes many minutes even when everything goes smoothly.   A value for the environment variable \texttt{PETSC_ARCH} will be reported at the end of the configure process; take note of this value.  (Note that a previously installed PETSc can be reconfigured with a new \texttt{PETSC_ARCH} if necessary.)

\item  After \texttt{configure.py} finishes, you will need to \texttt{make all test} in the PETSc directory and watch the result.  If the X Windows system is functional some example viewers will appear; as noted you will need the X header files for this to work.

\item Finally, you will want to set the \texttt{PETSC_DIR} and the \texttt{PETSC_ARCH} environment variables in your \texttt{.profile} or \texttt{.bashrc} file.  Also remember to add the MPI \texttt{bin} directory to your \texttt{PATH}.  For instance, if you used the option \texttt{--download-mpich=1} in the PETSc configure, the MPI \texttt{bin} directory will have a path like \texttt{\$PETSC_DIR/} \texttt{externalpackages/mpich2-1.0.4p1/\$PETSC_ARCH/bin/}.  Therefore the lines 
\begin{verbatim}
export PETSC_DIR=/home/user/petsc-2.3.3-p2/
export PETSC_ARCH=linux-gnu-cxx
export PATH=$PETSC_DIR/externalpackages/mpich2-1.0.4p1/$PETSC_ARCH/bin/:$PATH
\end{verbatim}
\noindent could appear in one of those files.
\end{enumerate}
\end{enumerate}
\medskip See the table in the section \ref{sec:prerequisites} for a summary of the dependencies on external libraries, including
those mentioned so far.

\subsubsection{Installing PISM itself}
\label{sec:install-cmake}
At this point you have configured the environment which PISM needs.  You are ready to build PISM itself, which is a much quicker procedure!

\begin{enumerate}\setcounter{enumi}{7}
\item Get the latest source\index{PISM!download source code} for PISM using the Subversion\index{Subversion} version control system:
\begin{enumerate}
\item \label{getPISMstep} Check the website \url{http://www.pism-docs.org/} for the latest version of PISM.
\item Do
\begin{verbatim}
$  svn co http://svn.gna.org/svn/pism/tags/stable0.4 pism0.4
\end{verbatim}
\item A directory called ``\texttt{pism0.4/}'' will be created.  Note that in the future when you enter that directory,
  \texttt{svn update} will update to the latest revision of PISM.\footnote{Of course, after \t{svn update} you will \t{make -C
      build install} to recompile and re-install PISM.}
\end{enumerate}

\item Make sure that \texttt{PETSC_ARCH} and \texttt{PETSC_DIR} variables are set.

\item Build PISM:\footnote{Please report any problems you meet at this build stage by sending us the output: \href{mailto:help@pism-docs.org}{help@pism-docs.org}.}
\begin{verbatim}
$  mkdir pism0.4/build
$  cd pism0.4/build
$  PISM_INSTALL_PREFIX=~/pism cmake ..
$  make install
\end{verbatim}%$

  This will configure PISM to be installed in \texttt{\textasciitilde/pism/bin}, \texttt{\textasciitilde/pism/lib/pism} and \texttt{\textasciitilde/pism/share/doc}, then
  compile and install all its executables and scripts.

If your operating system does not support shared libraries\footnote{This might be necessary if you're building on a Cray XT5 or a Sun Opteron Cluster, for example.}, then set \texttt{BUILD_SHARED_LIBS} to ``OFF''. This can be done by
either running
\begin{verbatim}
$  cmake -DBUILD_SHARED_LIBS:BOOL=OFF ..
\end{verbatim}%$
or by using \texttt{ccmake}: run
\begin{verbatim}
$  ccmake ..
\end{verbatim}%$
and then change \texttt{BUILD_SHARED_LIBS} (press 'c' (configure), then 'g' (generate makefiles))
before \texttt{make install}.

Object files created during the build process (located in the \texttt{build} sub-directory) are not deleted after installing PISM,
so do ``\texttt{make clean}'' to delete these if space is an issue. You can also delete the build directory if you are not
planning on re-compiling PISM.

\item PISM executables can be run most easily by adding the \texttt{bin/} sub-directory in your selected install path
  (\texttt{\textasciitilde/pism/bin} in the example above) to your \texttt{PATH}.  For instance, this command can be done in the \texttt{bash} shell or in your \texttt{.bashrc} file:\index{setting the \$PATH to find PISM executables}
\begin{verbatim}
export PATH=~/pism/bin:$PATH
\end{verbatim}

\item See the \emph{Getting Started} section \emph{User's Manual} to continue.
\end{enumerate}

\clearpage
\subsection{Installing PISM on Mac OS X}  \label{subsec:macosx}

Please refer to section \ref{subsec:generic} for the description of PISM's use of prerequisites and generic installation
instructions. This section describes an adaptation of the same process to the Mac OS X operating system.

\begin{enumerate}
\item As PISM is distributed as compilable source code only, you will need software developer's tools, XCode and the \emph{X
    window system}, X11. Both packages can be installed by either downloading them from
  \href{http://developer.apple.com/tools/xcode/}{Apple Developer Connection} or using the Mac OS X installation DVD.
\item The use of \href{http://www.macports.org/}{MacPorts} or \href{http://www.finkproject.org/}{Fink} is recommended, as it significantly simplifies installing many
  open-source libraries. Download a package from the \href{http://www.macports.org/install.php}{MacPorts homepage} (or
  (\href{http://www.finkproject.org/download/index.php}{Fink homepage}, install and set the environment:

\begin{verbatim}
export PATH=/opt/local/bin:/opt/local/sbin:$PATH
\end{verbatim}
for MacPorts and
\begin{verbatim}
source /sw/bin/init.sh
\end{verbatim}
for Fink.

\item Note that it is not necessary to install Python, as it is bundled with the operating system (as are BLAS, LAPACK, OpenMPI
  and Subversion).

  Some PISM scripts use Scipy. It is possible to install it using MacPorts,  using \href{http://www.enthought.com/}{Enthought Python Distributions} is another option.

\item  If you're using MacPorts, do
\begin{verbatim}
$  sudo port install netcdf ncview gsl fftw-3
\end{verbatim}%$
to install NetCDF, ncview, GSL and FFTW.

If you're using Fink, use the following command instead (\texttt{ncview} is only available in the unstable branch).
\begin{verbatim}
$  fink install netcdf gsl fftw3
\end{verbatim}

\item At this point, all the PISM prerequisites except PETSc are installed. Download the latest PETSc tarball from the
  \href{http://www.mcs.anl.gov/petsc/petsc-as/}{PETSc website}. Untar, then change to the directory just created.
The next three commands completed the PETSc installation:
\begin{verbatim}
$  export PETSC_DIR=$PWD; export PETSC_ARCH=macosx;
$  ./config/configure.py --with-shared --with-dynamic --with-debugging=no \
        --with-fortran=0 --with-blas-lapack-dir=/usr/ --with-mpi-dir=/usr/
$  make all test
\end{verbatim}
\item Finally, to build PISM, see section \ref{sec:install-cmake}.

\item See the \emph{Getting Started} section \emph{User's Manual} to continue.
\end{enumerate}

\clearpage
\subsection{If it still does not work}
\label{subsec:config}

We recommend using \texttt{ccmake}, the text-based CMake interface to adjust PISM's build parameters. One can also set CMake cache
variables using the \texttt{-D} command-line option (\texttt{cmake -Dvariable=value}) or by editing \texttt{CMakeCache.txt} in the
build directory.

Here are some issues we know about.
\begin{itemize}
\item Sometimes it is necessary to set \texttt{GSL_CBLAS_LIB} using \texttt{cmake -Dvariable=value} mechanism if installation
  fails at the linking stage.
\item Sometimes if a system has more than one MPI installation CMake finds the wrong one. To tell it which one to use, set
  \texttt{MPI_LIBRARY} and related variables by using \texttt{ccmake} or editing \texttt{CMakeCache.txt} in the build directory.

  It is also possible to guide CMake's configuration mechanism by setting \texttt{MPI_COMPILER} to the compiler (such as
  \texttt{mpicc}) corresponding to the MPI installation you want to use, setting \texttt{MPI_LIBRARY} to
  \texttt{MPI_LIBRARY-NOTFOUND} and re-running CMake.
\item If you are compiling PISM on a system using a cross-compiler, you will need to disable CMake's tests trying to determine if
  PETSc is installed properly. To do this, set \texttt{PETSC_EXECUTABLE_RUNS} to ``yes''.

  To tell CMake where to look for libraries for the target system, see \url{http://www.cmake.org/Wiki/CMake_Cross_Compiling} and
  the paragraph about \texttt{CMAKE_FIND_ROOT_PATH} in particular.
\end{itemize}



\clearpage
\section{Quick tests of the installation}
Once you're done with the installation, a few tests can confirm that PISM is functioning correctly.
\begin{enumerate}
\item Try a MPI four process verification run:

\begin{verbatim}
$ mpiexec -n 4 pismv -test G -y 200
\end{verbatim}

\noindent If you see some output and a final \texttt{Writing model state} \texttt{to file 'verify.nc'} \texttt{... done} then PISM
completed successfully.  At the end of this run you get measurements of the difference between the numerical result and the exact solution.  See the \emph{User's Manual} for more on PISM verification.

The above ``\texttt{-n 4}'' run should work even if there is only one actual processor (core) on your machine, in which case MPI will just run multiple processes on the one processor.  This run will also produce a NetCDF output file \texttt{verify.nc}, which can be read and viewed by NetCDF tools.

\item Try an EISMINT II run using the PETSc viewers (under the X window system):

\begin{verbatim}
$ pisms -y 5000 -view_map thk,temppabase,csurf
\end{verbatim}

\noindent When using such viewers and \texttt{mpiexec} the additional final option \texttt{-display :0} \,is sometimes required to enable MPI to use X, like this:

\begin{verbatim}
$ mpiexec -n 2 pisms -y 5000 -view_map thk,temppabase,csurf -display :0
\end{verbatim}

\item Run a basic suite of software tests: \footnote{Note that these tests use NCO and depend on Python packages \texttt{numpy}
    and \texttt{netcdf4-python}}
\begin{verbatim}
$ make test
\end{verbatim}
(You need to be in PISM's build directory.)

\noindent Feel free to email us about failed tests, etc.: \href{mailto:help@pism-docs.org}{help@pism-docs.org}.
\end{enumerate}

\subsubsection*{Next steps}  Start with the \emph{User's Manual}, which has a ``Getting started'' section.  A copy is on-line at the PISM homepage and documentation page \href{http://www.pism-docs.org/}{\t{www.pism-docs.org}}, along with a source code \emph{Browser} (HTML).  Completely up-to-date documentation can be built from \LaTeX~source in the \texttt{doc/} sub-directory, as described in the last section.

A final reminder with respect to installation:  Let's assume you have checked out a copy of PISM using Subversion, as in step
\ref{getPISMstep} above.  You can update your copy of PISM to the latest version by running \texttt{svn up} in the PISM directory
and \texttt{make install} in your build directory.


\newpage
\section{Installing Python packages}
\label{sec:python}

If you're lucky, you might be able to install all the Python packages mentioned in section \ref{sec:prerequisites} using a package manager. On the other hand, the Python packages below are not currently available in Debian package repositories. They are easy to install using Python \texttt{setuptools}, however; these tools are included with recent versions of Python.  And you might not be using a package manager!

\subsubsection*{Python module \texttt{netCDF3}, from package \texttt{netcdf4-python}}  You can skip this paragraph if you have
\href{http://www.enthought.com/}{Enthought Python Distributions} installed.

If you have NetCDF 3.6.2, you will need to download version 0.9.3 of \texttt{netcdf4-python} (later versions do not have the
\texttt{netCDF3} module). Recent versions require HDF5 1.8.6 and NetCDF-4 4.1.2.

To install \texttt{netcdf4-python} providing get \texttt{netCDF3} or \texttt{netCDF4} modules needed by PISM scripts, download a
tarball from the project homepage \url{http://code.google.com/p/netcdf4-python/}.

\begin{verbatim}
$  wget http://netcdf4-python.googlecode.com/files/netCDF4-NUMBER.tar.gz
$  tar -xvf netCDF4-NUMBER.tar.gz
\end{verbatim}
Enter the directory you just untarred and install:
\begin{verbatim}
$  cd netCDF4-NUMBER/
$  sudo NETCDF3_DIR=/usr/ python setup-nc3.py install
\end{verbatim}
assuming that NetCDF was installed in the \texttt{/usr/} tree.

Under Mac OS X,
\begin{verbatim}
$  sudo NETCDF3_DIR=/opt/local/ python setup-nc3.py install
\end{verbatim}

If you actually use NetCDF 4.0,
\begin{verbatim}
$  sudo NETCDF4_DIR=/usr/ HDF5_DIR=/usr/ python setup.py install
\end{verbatim}%$
assuming that HDF5 and NetCDF were installed in the \texttt{/usr/} tree.

The scripts in directories \texttt{util/}, \texttt{examples/eisgreen}, \texttt{examples/eisross}, \texttt{examples/mismip}, and so
on, which need \texttt{netCDF3} or \texttt{netCDF4}, should now work.


\subsubsection*{Python package \texttt{scikits.delaunay}}  One PISM script, \texttt{examples/eisross/ross_plot.py}, needs this. \href{http://www.enthought.com/}{Enthought Python Distributions} users do (make sure you have version 6.0 or greater): 
\begin{verbatim}
$ sudo easy_install http://svn.scipy.org/svn/scikits/trunk/delaunay
\end{verbatim}
If you have a different Python distribution installed, here's how to get \texttt{scikits.delaunay}:
\begin{verbatim}
$  svn co http://svn.scipy.org/svn/scikits/trunk/delaunay
$  cd delaunay
$  sudo python setup.py install
\end{verbatim}
(The above use of \texttt{setup.py} will work if the python ``\texttt{setuptools}'' are installed.  On a Debian, that may require getting the \texttt{python-setuptools} package, for example.)

\newpage
\section{Rebuilding PISM documentation}
\label{sec:docs}

You might want to rebuild the documentation from source, as PISM and its documentation evolve together.  These tools are required:
\bigskip
\begin{center}
  \begin{tabular*}{0.9\linewidth}{llp{0.55\linewidth}}
    \toprule
    \LaTeX & \href{http://www.latex-project.org/}{\t{www.latex-project.org}} &  needed for rebuilding any of the documentation \\
    \texttt{doxygen}\index{doxygen} & \href{http://www.stack.nl/~dimitri/doxygen/}{\t{www.doxygen.org}} &  for rebuilding the \emph{Browser} from source  \\
    \texttt{graphviz}\index{graphviz} & \href{http://www.graphviz.org/}{\t{www.graphviz.org}} &  used in diagrams in the \emph{Browser}  \\
    \bottomrule
  \end{tabular*}
\end{center}
\bigskip
\noindent To rebuild PISM documentation, change to the PISM build directory and do
\begin{center}
  \begin{tabular}{p{0.22\linewidth}p{0.75\linewidth}}
    \texttt{make userman} & to build the \emph{User's Manual}, \texttt{manual.pdf}\\
    \texttt{make installation} & to build this document, the \emph{Installation Manual}, \texttt{installation.pdf}\\
    \texttt{make browser_base} & to build the \emph{PISM Source Code Browser},\\
    &\texttt{doc/browser/base/html/index.html}\\
    \texttt{make browser_util} & to build the \emph{PISM Utility Class Browser},\\
    &\texttt{doc/browser/util/html/index.html}\\
    \texttt{make cheatsheet} & to build the 2-page PISM Cheat-Sheet summarizing its command-line options\\
  \end{tabular}
\end{center}

\end{document}

%%% Local Variables:
%%% fill-column: 130
%%% End: 
