\documentclass{article}
\usepackage[margin=1in]{geometry}
\usepackage{amsmath}

\begin{document}

\title{Melting parametrization for PISM}


\author{Matthias Mengel and Johannes Feldmann}

\maketitle

\begin{abstract}
We derive the melting parametrization for PISM, based on the input of potential temperature and salinity from an ocean model. The conversion from potential to in situ temperature is not needed anymore due to the use of a linearized freezing point equation for potential temperature.


\end{abstract}

\newpage

\section{Melting parametrization}
\label{sec_meltparam}


The melting and refreezing process in the brine layer requires exchange fluxes with the open ocean and the ice shelf for mass, energy and salinity.
We use indices $i,m,b$ for the ice, mixed ocean layer and brine layer respectively.
Energy and salinity conservation requires
\begin{align*}
0 &= Qsi + Qsm + Qsb \\
0 &= Qti + Qtm + Qtb
\end{align*}
With positive fluxes pointing towards the brine layer and sources with positive sign.
 %as in Fig. 1 of holland\_jenkins99.
Salt flux to ice $Qsi$ is zero. The heat flux at the ice-brine interface is
\begin{equation}
Qti = -\rho_i \ cp_i \ \kappa \ \nabla T
\end{equation}
We follow the \cite{Wexler1960,FoldvikKvinge1974,Hellmeretal1998}
definition of $\nabla T$ at the base of the ice shelf that incorporates the need to warm the ice before it is melting.
\begin{equation*}
\nabla T = \Delta T \frac{wb}{\kappa}
\end{equation*}
And thus
\begin{equation}
Qti = - \rho_i \ cp_i \ wb \ (Tb - Ts)
\end{equation}
with $\Delta T = Tb - Ts$ as the temperature difference between the brine layer and the surface of the ice shelf.
The source/sink of heat and salt in the brine layer is due to melting/refreezing.
$wb$ is freshwater production or melt rate with $wb>0$ as melting and $wb<0$ refreezing, $Lf$ is latent heat and $\rho_i$ is the ice density.
%and $\rho_m$ is ocean mixed layer reference density.
%NOTE: this differs from holland_jenkins99, who use $\rho_m$ in the following equation.

\begin{align}
Qtb &= -\rho_i \ wb \ Lf \\
Qsb &= \rho_i \ wb \ (Si-Sb) = -\rho_i \ wb \ Sb
%Qtb = -\rho_i*wb*Lf \\
%Qsb = -\rho_m*wb*(Sb-Si)
\end{align}
With the salinity in the ice $Si$ being approximately zero.
The parametrized (turbulent) diffusion of heat/salt from ocean mixed layer with
$cp_m$ as ocean mixed layer heat capacity, gat and gas as exchange velocities are
\begin{align}
Qtm &= \rho_m \ cp_m \ gat \ (Tm-Tb) \\
Qsm &= \rho_m \ gas \ (Sm-Sb)
\end{align}

We assume that the in situ temperature difference (Tm-Tb) is the same for its corresponding potential temperature counterparts, therefore:
\begin{equation}
Qtm = \rho_m \ cp_m \ gat \ (\Theta m- \Theta b)
\end{equation}

We use two linearized equations for the freezing point, for in situ temperature and potential temperature, respectively.
The linearizations are a fit to the more complex function of the freezing point that is in general:
\begin{equation}
Tb  = f(salinity,pressure)
\end{equation}
In our case a linearized version is appropriate because the boundary layer temperature and salinity will not deviate far from the freezing conditions.
The general equation has been approximated for a larger salinity and temperature range by \cite{Jackettetal2006} for example.

We write the in-situ linearization as
\begin{equation}
Tb  = ai*Sb + bi + ci*p
\end{equation}
with the coefficients adapted from \cite{Hellmeretal1998} that go back to \cite{FoldvikKvinge1974}: $ai=-0.0575 ^\circ C/psu$, $bi=0.0901 ^\circ C$ and $ci = -7.61e-4 ^\circ C/m$.
To avoid the conversion of potential ocean temperature to in situ temperature we apply a second linearization with respect to potential temperature:
\begin{equation}
\Theta b  = ap*Sb + bp + cp*p
\end{equation}
Values are $ap=-0.0575 ^\circ C/psu$, $bp=0.09212 ^\circ C$ and $cp = -7.852e-4 ^\circ C/m$.
We determine the coefficients by converting in situ temperatures to potential temperatures with the known function $g$:
\begin{equation}
\Theta = g (T_{insitu})
\end{equation}
and then fit
\begin{equation}
\Theta b= g (T b)
\end{equation}
with fixed coefficient for salinity $ap=ai$. For consistency we use $g$ as defined in the melting code of FESOM \cite{Wangetal2013}. The melting code applied here has been derived from the FESOM routines.
Thank Xylar Asay-Davis (for the suggestion on two linearizations for the freezing point equation that lead to a major simplification of the code).


\subsection{derive}
From heat flux balance we get
\begin{align}
0 &= Qti + Qtm + Qtb \\
0 &= \rho_i \ cp_i \ wb \ (Ts - Tb) + \rho_m \ cp_m \ gat \ (\Theta m- \Theta b) - \rho_i \ wb \ Lf
\label{eq_hbalance}
\end{align}
From salinity balance with salt flux from ice $Qsi = 0$ we get
\begin{align*}
0 &= Qsm + Qsb \\
0 &= - \rho_m \ gas \ (Sm-Sb) -\rho_i \ wb \ Sb \\
wb &= \frac{\rho_m \ gas \ (Sm-Sb)}{\rho_i \ Sb}
\end{align*}
We can insert $wb$ into equation \ref{eq_hbalance} and get
\begin{align}
0 &= wb \ (\rho_i \ cp_i \ (Ts - Tb) -  \rho_i \ Lf) + \rho_m \ cp_m \ gat \ (\Theta m- \Theta b) \\
0 &= \frac{\rho_m \ gas \ (Sm-Sb)}{Sb} \ (cp_i \ (Ts - Tb) -  Lf) + \rho_m \ cp_m \ gat \ (\Theta m- \Theta b)
\end{align}
Substituting $Tb$ yields a quadratic equation in brine layer salinity $Sb$
\begin{align*}
0 &= \rho_m \ gas \ (Sm-Sb) (cp_i \ (Ts - Tb) -  Lf) + Sb \ \rho_m \ cp_m \ gat \ (\Theta m- \Theta b) \\
0 &= gas \ (Sm-Sb)(cp_i \ (Ts - a \ Sb - b - c \ p) -  Lf) + Sb \ cp_m \ gat \ (\Theta m - d \ Sb - e - f \ p)
\end{align*}
Sorting as a polynomial in $Sb$ gives
\begin{align*}
0 = & Sb^2 \ (a \ cp_i \ gas - d \ cp_m \ gat) \ + \\
& Sb \ (cp_i \ gas \ (b + c \ p - Ts - Sm \ a) +gas \ Lf + cp_m \ gat(\Theta m-e-f \ p)) + \\
& gas \ Sm \ ( cp_i(Ts -b - c \ p) - Lf )
\end{align*}


The coefficients of the polynomial will correspond to $ex1$ and $ex2$ that is solved in the 3eqn melting parametrization \texttt{POGivenTH::btemp\_bmelt\_3eqn} in PISM.
The conversion of potential to in situ temperature will not be needed any longer.

\bibliographystyle{siam}
\bibliography{bib_meltingparam}

\end{document}
