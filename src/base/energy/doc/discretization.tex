\documentclass{article}
\usepackage{amsmath}
\usepackage{hyperref}
\usepackage[left=0.5in,right=0.5in]{geometry}
\parindent=0in \parskip=0.5\baselineskip

\newcommand{\lhsI}{
&\displaystyle -{{2\,w_{i+{{1}\over{2}},j}\,N_{i+{{1}\over{2}},j}\,
 \left({{\Delta_y\left(v_{i+1,j}\right)+\Delta_y\left(v_{i,j}\right)
 }\over{4\,\Delta y}}+{{2\,\delta_{+x}\left(u_{i,j}\right)}\over{
 \Delta x}}\right)}\over{\Delta x}} & \\
&\displaystyle -{{w_{i,j+{{1}\over{2}}}\,N_{i,j+{{1}\over{2}}}\,
 \left({{\Delta_x\left(v_{i,j+1}\right)+\Delta_x\left(v_{i,j}\right)
 }\over{4\,\Delta x}}+{{\delta_{+y}\left(u_{i,j}\right)}\over{
 \Delta y}}\right)}\over{\Delta y}} & \\
&\displaystyle {{2\,w_{i-{{1}\over{2}},j}\,N_{i-{{1}\over{2}},j}\,
 \left({{\Delta_y\left(v_{i,j}\right)+\Delta_y\left(v_{i-1,j}\right)
 }\over{4\,\Delta y}}+{{2\,\delta_{-x}\left(u_{i,j}\right)}\over{
 \Delta x}}\right)}\over{\Delta x}} & \\
&\displaystyle {{w_{i,j-{{1}\over{2}}}\,N_{i,j-{{1}\over{2}}}\,\left(
 {{\Delta_x\left(v_{i,j}\right)+\Delta_x\left(v_{i,j-1}\right)}\over{
 4\,\Delta x}}+{{\delta_{-y}\left(u_{i,j}\right)}\over{\Delta y}}
 \right)}\over{\Delta y}} & \\
}
\newcommand{\lhsII}{
&\displaystyle -{{w_{i+{{1}\over{2}},j}\,N_{i+{{1}\over{2}},j}\,
 \left({{\Delta_y\left(u_{i+1,j}\right)+\Delta_y\left(u_{i,j}\right)
 }\over{4\,\Delta y}}+{{\delta_{+x}\left(v_{i,j}\right)}\over{
 \Delta x}}\right)}\over{\Delta x}} & \\
&\displaystyle -{{2\,w_{i,j+{{1}\over{2}}}\,N_{i,j+{{1}\over{2}}}\,
 \left({{\Delta_x\left(u_{i,j+1}\right)+\Delta_x\left(u_{i,j}\right)
 }\over{4\,\Delta x}}+{{2\,\delta_{+y}\left(v_{i,j}\right)}\over{
 \Delta y}}\right)}\over{\Delta y}} & \\
&\displaystyle {{2\,w_{i,j-{{1}\over{2}}}\,N_{i,j-{{1}\over{2}}}\,
 \left({{2\,\delta_{-y}\left(v_{i,j}\right)}\over{\Delta y}}+{{
 \Delta_x\left(u_{i,j}\right)+\Delta_x\left(u_{i,j-1}\right)}\over{4
 \,\Delta x}}\right)}\over{\Delta y}} & \\
&\displaystyle {{w_{i-{{1}\over{2}},j}\,N_{i-{{1}\over{2}},j}\,\left(
 {{\delta_{-x}\left(v_{i,j}\right)}\over{\Delta x}}+{{\Delta_y\left(u
 _{i,j}\right)+\Delta_y\left(u_{i-1,j}\right)}\over{4\,\Delta y}}
 \right)}\over{\Delta x}} & \\
}
\newcommand{\lhsIII}{
&\displaystyle -{{2\,w_{i+{{1}\over{2}},j}\,N_{i+{{1}\over{2}},j}\,
 \left({{w_{i+1,j-{{1}\over{2}}}\,\delta_{{\it -y}}(v_{i+1,j})+w_{i+1
 ,j+{{1}\over{2}}}\,\delta_{{\it +y}}(v_{i+1,j})+w_{i,j-{{1}\over{2}}
 }\,\delta_{{\it -y}}(v_{i,j})+w_{i,j+{{1}\over{2}}}\,\delta_{
 {\it +y}}(v_{i,j})}\over{4\,\Delta y}}+{{2\,w_{i+{{1}\over{2}},j}\,
 \delta_{{\it +x}}(u_{i,j})}\over{\Delta x}}\right)}\over{\Delta x}} & \\
&\displaystyle -{{w_{i,j+{{1}\over{2}}}\,N_{i,j+{{1}\over{2}}}\,
 \left({{w_{i-{{1}\over{2}},j+1}\,\delta_{{\it -x}}(v_{i,j+1})+w_{i+
 {{1}\over{2}},j+1}\,\delta_{{\it +x}}(v_{i,j+1})+w_{i-{{1}\over{2}},
 j}\,\delta_{{\it -x}}(v_{i,j})+w_{i+{{1}\over{2}},j}\,\delta_{
 {\it +x}}(v_{i,j})}\over{4\,\Delta x}}+{{w_{i,j+{{1}\over{2}}}\,
 \delta_{{\it +y}}(u_{i,j})}\over{\Delta y}}\right)}\over{\Delta y}} & \\
&\displaystyle {{2\,w_{i-{{1}\over{2}},j}\,N_{i-{{1}\over{2}},j}\,
 \left({{w_{i,j-{{1}\over{2}}}\,\delta_{{\it -y}}(v_{i,j})+w_{i,j+{{1
 }\over{2}}}\,\delta_{{\it +y}}(v_{i,j})+w_{i-1,j-{{1}\over{2}}}\,
 \delta_{{\it -y}}(v_{i-1,j})+w_{i-1,j+{{1}\over{2}}}\,\delta_{
 {\it +y}}(v_{i-1,j})}\over{4\,\Delta y}}+{{2\,w_{i-{{1}\over{2}},j}
 \,\delta_{{\it -x}}(u_{i,j})}\over{\Delta x}}\right)}\over{\Delta x
 }} & \\
&\displaystyle {{w_{i,j-{{1}\over{2}}}\,N_{i,j-{{1}\over{2}}}\,\left(
 {{w_{i-{{1}\over{2}},j}\,\delta_{{\it -x}}(v_{i,j})+w_{i+{{1}\over{2
 }},j}\,\delta_{{\it +x}}(v_{i,j})+w_{i-{{1}\over{2}},j-1}\,\delta_{
 {\it -x}}(v_{i,j-1})+w_{i+{{1}\over{2}},j-1}\,\delta_{{\it +x}}(v_{i
 ,j-1})}\over{4\,\Delta x}}+{{w_{i,j-{{1}\over{2}}}\,\delta_{{\it -y}
 }(u_{i,j})}\over{\Delta y}}\right)}\over{\Delta y}} & \\
}
\newcommand{\CUfirstInterior}{
&-1 & 0 & 1 \\\hline
\hline
$1$
&$\displaystyle 0$
&$\displaystyle -{{c_{n}}\over{\Delta y^2}}$
&$\displaystyle 0$
\\
\hline
$0$
&$\displaystyle -{{4\,c_{w}}\over{\Delta x^2}}$
&$\displaystyle {{c_{s}+c_{n}}\over{\Delta y^2}}+{{4\,\left(c_{w}+
 c_{e}\right)}\over{\Delta x^2}}$
&$\displaystyle -{{4\,c_{e}}\over{\Delta x^2}}$
\\
\hline
$-1$
&$\displaystyle 0$
&$\displaystyle -{{c_{s}}\over{\Delta y^2}}$
&$\displaystyle 0$
\\
}
\newcommand{\CUsecondInterior}{
&-1 & 0 & 1 \\\hline
\hline
$1$
&$\displaystyle {{c_{w}+2\,c_{n}}\over{4\,\Delta x\,\Delta y}}$
&$\displaystyle {{c_{w}-c_{e}}\over{4\,\Delta x\,\Delta y}}$
&$\displaystyle -{{2\,c_{n}+c_{e}}\over{4\,\Delta x\,\Delta y}}$
\\
\hline
$0$
&$\displaystyle -{{c_{s}-c_{n}}\over{2\,\Delta x\,\Delta y}}$
&$\displaystyle 0$
&$\displaystyle {{c_{s}-c_{n}}\over{2\,\Delta x\,\Delta y}}$
\\
\hline
$-1$
&$\displaystyle -{{c_{w}+2\,c_{s}}\over{4\,\Delta x\,\Delta y}}$
&$\displaystyle -{{c_{w}-c_{e}}\over{4\,\Delta x\,\Delta y}}$
&$\displaystyle {{2\,c_{s}+c_{e}}\over{4\,\Delta x\,\Delta y}}$
\\
}
\newcommand{\CVfirstInterior}{
&-1 & 0 & 1 \\\hline
\hline
$1$
&$\displaystyle {{2\,c_{w}+c_{n}}\over{4\,\Delta x\,\Delta y}}$
&$\displaystyle {{c_{w}-c_{e}}\over{2\,\Delta x\,\Delta y}}$
&$\displaystyle -{{c_{n}+2\,c_{e}}\over{4\,\Delta x\,\Delta y}}$
\\
\hline
$0$
&$\displaystyle -{{c_{s}-c_{n}}\over{4\,\Delta x\,\Delta y}}$
&$\displaystyle 0$
&$\displaystyle {{c_{s}-c_{n}}\over{4\,\Delta x\,\Delta y}}$
\\
\hline
$-1$
&$\displaystyle -{{2\,c_{w}+c_{s}}\over{4\,\Delta x\,\Delta y}}$
&$\displaystyle -{{c_{w}-c_{e}}\over{2\,\Delta x\,\Delta y}}$
&$\displaystyle {{c_{s}+2\,c_{e}}\over{4\,\Delta x\,\Delta y}}$
\\
}
\newcommand{\CVsecondInterior}{
&-1 & 0 & 1 \\\hline
\hline
$1$
&$\displaystyle 0$
&$\displaystyle -{{4\,c_{n}}\over{\Delta y^2}}$
&$\displaystyle 0$
\\
\hline
$0$
&$\displaystyle -{{c_{w}}\over{\Delta x^2}}$
&$\displaystyle {{4\,\left(c_{s}+c_{n}\right)}\over{\Delta y^2}}+{{
 c_{w}+c_{e}}\over{\Delta x^2}}$
&$\displaystyle -{{c_{e}}\over{\Delta x^2}}$
\\
\hline
$-1$
&$\displaystyle 0$
&$\displaystyle -{{4\,c_{s}}\over{\Delta y^2}}$
&$\displaystyle 0$
\\
}
\newcommand{\CUfirstMargin}{
&-1 & 0 & 1 \\\hline
\hline
$1$
&$\displaystyle 0$
&$\displaystyle -{{c_{n}\,{\it bPP}^2}\over{\Delta y^2}}$
&$\displaystyle 0$
\\
\hline
$0$
&$\displaystyle -{{4\,c_{w}\,{\it aMM}^2}\over{\Delta x^2}}$
&$\displaystyle {{c_{n}\,{\it bPP}^2+c_{s}\,{\it bMM}^2}\over{
 \Delta y^2}}+{{4\,\left(c_{e}\,{\it aPP}^2+c_{w}\,{\it aMM}^2\right)
 }\over{\Delta x^2}}$
&$\displaystyle -{{4\,c_{e}\,{\it aPP}^2}\over{\Delta x^2}}$
\\
\hline
$-1$
&$\displaystyle 0$
&$\displaystyle -{{c_{s}\,{\it bMM}^2}\over{\Delta y^2}}$
&$\displaystyle 0$
\\
}
\newcommand{\CUsecondMargin}{
C^{u,2}_{-1,-1} &=& \displaystyle -{{c_{w}\,{\it aMM}\,{\it bMw}}\over{4\,\Delta x\,\Delta y}} \\
&+& \displaystyle -{{c_{s}\,{\it aMs}\,{\it bMM}}\over{2\,\Delta x\,\Delta y}} \\
C^{u,2}_{-1,0} &=& \displaystyle {{c_{w}\,{\it aMM}\,\left({\it bMw}-{\it bPw}\right)}\over{4\,\Delta x\,\Delta y}} \\
&+& \displaystyle {{{\it aMM}\,\left(c_{n}\,{\it bPP}-c_{s}\,{\it bMM}\right)}\over{2\,\Delta x\,\Delta y}} \\
C^{u,2}_{-1,1} &=& \displaystyle {{c_{w}\,{\it aMM}\,{\it bPw}}\over{4\,\Delta x\,\Delta y}} \\
&+& \displaystyle {{c_{n}\,{\it aMn}\,{\it bPP}}\over{2\,\Delta x\,\Delta y}} \\
C^{u,2}_{0,-1} &=& \displaystyle {{c_{s}\,\left({\it aMs}-{\it aPs}\right)\,{\it bMM}}\over{2\,\Delta x\,\Delta y}} \\
&+& \displaystyle {{\left(c_{e}\,{\it aPP}-c_{w}\,{\it aMM}\right)\,{\it bMM}}\over{4\,\Delta x\,\Delta y}} \\
C^{u,2}_{0,0} &=& \displaystyle {{c_{n}\,\left({\it aPP}-{\it aMM}\right)\,{\it bPP}+c_{s}\,\left({\it aMM}-{\it aPP}\right)\,{\it bMM}}\over{2\,\Delta x\,\Delta y}} \\
&+& \displaystyle {{\left(c_{e}\,{\it aPP}-c_{w}\,{\it aMM}\right)\,{\it bPP}+\left(c_{w}\,{\it aMM}-c_{e}\,{\it aPP}\right)\,{\it bMM}}\over{4\,\Delta x\,\Delta y}} \\
C^{u,2}_{0,1} &=& \displaystyle {{c_{n}\,\left({\it aPn}-{\it aMn}\right)\,{\it bPP}}\over{2\,\Delta x\,\Delta y}} \\
&+& \displaystyle {{\left(c_{w}\,{\it aMM}-c_{e}\,{\it aPP}\right)\,{\it bPP}}\over{4\,\Delta x\,\Delta y}} \\
C^{u,2}_{1,-1} &=& \displaystyle {{c_{e}\,{\it aPP}\,{\it bMe}}\over{4\,\Delta x\,\Delta y}} \\
&+& \displaystyle {{c_{s}\,{\it aPs}\,{\it bMM}}\over{2\,\Delta x\,\Delta y}} \\
C^{u,2}_{1,0} &=& \displaystyle {{c_{e}\,{\it aPP}\,\left({\it bPe}-{\it bMe}\right)}\over{4\,\Delta x\,\Delta y}} \\
&+& \displaystyle {{{\it aPP}\,\left(c_{s}\,{\it bMM}-c_{n}\,{\it bPP}\right)}\over{2\,\Delta x\,\Delta y}} \\
C^{u,2}_{1,1} &=& \displaystyle -{{c_{e}\,{\it aPP}\,{\it bPe}}\over{4\,\Delta x\,\Delta y}} \\
&+& \displaystyle -{{c_{n}\,{\it aPn}\,{\it bPP}}\over{2\,\Delta x\,\Delta y}} \\
}
\newcommand{\CVfirstMargin}{
C^{v,1}_{-1,-1} &=& \displaystyle -{{c_{w}\,{\it aMM}\,{\it bMw}}\over{2\,\Delta x\,\Delta y}} \\
&+& \displaystyle -{{c_{s}\,{\it aMs}\,{\it bMM}}\over{4\,\Delta x\,\Delta y}} \\
C^{v,1}_{-1,0} &=& \displaystyle {{c_{w}\,{\it aMM}\,\left({\it bMw}-{\it bPw}\right)}\over{2\,\Delta x\,\Delta y}} \\
&+& \displaystyle {{{\it aMM}\,\left(c_{n}\,{\it bPP}-c_{s}\,{\it bMM}\right)}\over{4\,\Delta x\,\Delta y}} \\
C^{v,1}_{-1,1} &=& \displaystyle {{c_{w}\,{\it aMM}\,{\it bPw}}\over{2\,\Delta x\,\Delta y}} \\
&+& \displaystyle {{c_{n}\,{\it aMn}\,{\it bPP}}\over{4\,\Delta x\,\Delta y}} \\
C^{v,1}_{0,-1} &=& \displaystyle {{c_{s}\,\left({\it aMs}-{\it aPs}\right)\,{\it bMM}}\over{4\,\Delta x\,\Delta y}} \\
&+& \displaystyle {{\left(c_{e}\,{\it aPP}-c_{w}\,{\it aMM}\right)\,{\it bMM}}\over{2\,\Delta x\,\Delta y}} \\
C^{v,1}_{0,0} &=& \displaystyle {{c_{n}\,\left({\it aPP}-{\it aMM}\right)\,{\it bPP}+c_{s}\,\left({\it aMM}-{\it aPP}\right)\,{\it bMM}}\over{4\,\Delta x\,\Delta y}} \\
&+& \displaystyle {{\left(c_{e}\,{\it aPP}-c_{w}\,{\it aMM}\right)\,{\it bPP}+\left(c_{w}\,{\it aMM}-c_{e}\,{\it aPP}\right)\,{\it bMM}}\over{2\,\Delta x\,\Delta y}} \\
C^{v,1}_{0,1} &=& \displaystyle {{c_{n}\,\left({\it aPn}-{\it aMn}\right)\,{\it bPP}}\over{4\,\Delta x\,\Delta y}} \\
&+& \displaystyle {{\left(c_{w}\,{\it aMM}-c_{e}\,{\it aPP}\right)\,{\it bPP}}\over{2\,\Delta x\,\Delta y}} \\
C^{v,1}_{1,-1} &=& \displaystyle {{c_{e}\,{\it aPP}\,{\it bMe}}\over{2\,\Delta x\,\Delta y}} \\
&+& \displaystyle {{c_{s}\,{\it aPs}\,{\it bMM}}\over{4\,\Delta x\,\Delta y}} \\
C^{v,1}_{1,0} &=& \displaystyle {{c_{e}\,{\it aPP}\,\left({\it bPe}-{\it bMe}\right)}\over{2\,\Delta x\,\Delta y}} \\
&+& \displaystyle {{{\it aPP}\,\left(c_{s}\,{\it bMM}-c_{n}\,{\it bPP}\right)}\over{4\,\Delta x\,\Delta y}} \\
C^{v,1}_{1,1} &=& \displaystyle -{{c_{e}\,{\it aPP}\,{\it bPe}}\over{2\,\Delta x\,\Delta y}} \\
&+& \displaystyle -{{c_{n}\,{\it aPn}\,{\it bPP}}\over{4\,\Delta x\,\Delta y}} \\
}
\newcommand{\CVsecondMargin}{
&-1 & 0 & 1 \\\hline
\hline
$1$
&$\displaystyle 0$
&$\displaystyle -{{4\,c_{n}\,{\it bPP}^2}\over{\Delta y^2}}$
&$\displaystyle 0$
\\
\hline
$0$
&$\displaystyle -{{c_{w}\,{\it aMM}^2}\over{\Delta x^2}}$
&$\displaystyle {{4\,\left(c_{n}\,{\it bPP}^2+c_{s}\,{\it bMM}^2
 \right)}\over{\Delta y^2}}+{{c_{e}\,{\it aPP}^2+c_{w}\,{\it aMM}^2
 }\over{\Delta x^2}}$
&$\displaystyle -{{c_{e}\,{\it aPP}^2}\over{\Delta x^2}}$
\\
\hline
$-1$
&$\displaystyle 0$
&$\displaystyle -{{4\,c_{s}\,{\it bMM}^2}\over{\Delta y^2}}$
&$\displaystyle 0$
\\
}


\newcommand{\diff}[2]{\frac{\partial #1}{\partial #2}}

\begin{document}
\title{Discretization of the vertical problem for the conservation of energy}
\author{Constantine Khroulev}

\maketitle

\section{Introduction}
\label{sec:introduction}

These notes dress up \textsc{Maxima}-generated formulas needed to
implement (and check the implementation of) the energy balance code in
PISM, including the implementation of boundary conditions.

\begin{center}
  \begin{tabular}{ll}
    Symbol or formula & Comment \\
    \hline
    $\rho_{i}$ & ice density\\
    $c_{i}$ & specific heat capacity of ice\\
    $k_{i}$ & thermal conductivity of ice\\
    $\phi$ & heat flux at a boundary\\
    $G = \phi / K$ & enthalpy flux at a boundary (see also (\ref{eq:2}))\\
    $\Delta(U_{k}) = U_{k+1} - U_{k-1}$ & centered finite difference\\
    $\delta_{+}(U_{k}) = U_{k+1} - U_{k}$ & right-sided finite difference\\
    $\delta_{-}(U_{k}) = U_{k} - U_{k-1}$ & left-sided finite difference\\
    $\operatorname{Up}(U_{k}, \alpha) =
    \begin{cases}
      \delta_{+}(U_{k}), & \alpha < 0,\\
      \delta_{-}(U_{k}), & \alpha \ge 0.
    \end{cases}$ & first order upwinding\\
  \end{tabular}
\end{center}

\section{Energy balance in a column of ice}
\label{sec:energy-in-a-column}

The conservation of energy equation is

\begin{equation}
  \label{eq:1}
  \rho_{i} \cdot \frac{dE}{dt} = -\nabla\cdot\mathbf{q} + Q,
\end{equation}
where
$\mathbf{q}$, the enthalpy flux, is defined by
\begin{align}
  \label{eq:2}
  \mathbf{q} &= -K \nabla E,\\
  K(E) &=
  \begin{cases}
    k_{i}/c_{i}, & E \le E_{s},\\
    K_{0}, & E > E_{s}.\\
  \end{cases}
\end{align}
Here $E_{s}$ is the pressure-dependent enthalpy of the cold-temperate
transition and $Q$ is the source term consisting of the volumetric
strain heating and basal frictional heating.

Note that $K$ depeds on $E$, so the system defined by equations
(\ref{eq:1}) and (\ref{eq:2}) with appropriate boundary conditions is
nonlinear. The PISM implementation linearizes it by treating the time
derivative implicitly and using the \emph{previous} (or
initial) distribution of $E$ to evaluate $K(E)$.

PISM uses a shallow approximation of this equation without horizontal
conduction terms. Here I focus on the vertical direction, so
(\ref{eq:1}) becomes
\begin{equation}
  \label{eq:3}
  \rho_{i} \left( \diff{E}{t} + w\,\diff{E}{z} \right) - \diff{}{z}\left( K \diff{E}{z} \right) = \Phi.
\end{equation}
The term $\Phi$ above combines horizontal advection and the source:
\begin{equation}
  \label{eq:4}
    \Phi = Q - \rho_{i} \left( u\,\diff{E}{x} + v\,\diff{E}{y} \right).
\end{equation}

\section{Interior of the column}
\label{sec:interior}

The discretization scheme for equation (\ref{eq:3}) used in PISM is
\begin{equation}
  \label{eq:5}
  \discretization.
\end{equation}
Note the convex combination of centered finite difference and
first-order upwinding approximations of vertical advection with the
weight $\lambda$.

For $w \ge 0$ this can be rewritten as
\begin{equation}
  \label{eq:6}
  (\Lp)\, \El + (\Dp)\, \E + (\Up)\, \Eu = \Bp,
\end{equation}
and for $w < 0$
\begin{equation}
  \label{eq:7}
  (\Lm)\, \El + (\Dm)\, \E + (\Um)\, \Eu = \Bm.
\end{equation}

Here
\begin{equation}
  R^{n}_{k} = \R,\quad\quad \mu = \mufactor.
\end{equation}

This corresponds to the following lower-diagonal ($L_{k}$), diagonal
($D_{k}$) and upper-diagonal ($U_{k}$) entries in the tridiagonal
matrix corresponding to the discretization (\ref{eq:5}).

\newcommand{\wcases}[3]{#1 &= \begin{cases} #2, & w < 0,\\#3, & w \ge 0.\end{cases}}

\begin{align}
  \wcases{L_{k}}{\Lm}{\Lp}\\
  \notag\\
  \wcases{D_{k}}{\Dm}{\Dp}\\
  \notag\\
  \wcases{U_{k}}{\Um}{\Up}
\end{align}
The right hand side $b$ has elements
\begin{equation}
  \label{eq:8}
  b_{k} = \Bp
\end{equation}
in both cases.

\section{Neumann B. C.}
\label{sec:neumann-b-c}

In the case of the initial boundary value system consisting of
equation (\ref{eq:1}) combined with
\begin{align}
  \label{eq:9}
  \left.\frac{\partial E}{\partial z}\right|_{z = 0} &= G_{0}(t),\\
  E(z_{\text{surface}}) &= f(t)
\end{align}
the Neumann boundary conditions at the base are implemented by
combining the generic discretization equation (\ref{eq:5}) with the
equation (\ref{eq:10}) approximating (\ref{eq:9}) and eliminating
$E_{-1}$.
\begin{equation}
  \label{eq:10}
  \neumannb.
\end{equation}
This gives
\begin{align}
  \wcases{D_{0}}{\Dmb}{\Dpb}\\
  \notag\\
  \wcases{U_{0}}{\Umb}{\Upb}\\
  \notag\\
  \wcases{b_{0}}{\Bmb}{\Bpb}
\end{align}

When assembling the basal equation we approxomate $R_{-\frac{1}{2}}$ by $R_{0}$.

\newcommand{\ks}{k_{s}}

Similarly, when the Neumann B.~C. is imposed at the surface ($k =
\ks$), we combine (\ref{eq:5}) with (\ref{eq:11}) and eliminate $E_{\ks+1}$.
\begin{equation}
  \label{eq:11}
  \neumanns.
\end{equation}
This gives
\begin{align}
  \wcases{L_{\ks}}{\Lms}{\Lps}\\
  \notag\\
  \wcases{D_{\ks}}{\Dms}{\Dps}\\
  \notag\\
  \wcases{b_{\ks}}{\Bms}{\Bps}
\end{align}

When assembling the surface equation we approxomate $R_{\ks + \frac{1}{2}}$ by $R_{\ks}$.

\end{document}
